%!TEX root = paper.tex
%% FLIP’S MACROS FOR COMMENTS
%% USES: FlipPreamble.tex
%%
%% These macros are for communicating between collaborators
%% during the editing process.

%% USAGE SUMMARY: EXAMPLES
%% -----------------------
%% \comment{Check}{Is this equation correcT?}
%% \comment{Flip}{I think it is correct.}
%%
%% \begin{flipcomment}
%% This is a more involved comment, perhaps with equations
%% \end{flipcomment}
%%
%% \comment{Flip}{Can we discuss adding this:}
%% \new{This chunk of text is new compared to the last version}
%%
%% \comment{Flip}{Can we discuss removing this?}
%% \remove{Previous version of the text that I propose removing.} 



%% INLINE COMMENTS
%% ---------------
%% \comment is now multiply defined with \usepackage[most]{tcolorbox}
% \newcommand{\comment}[2]{\textcolor{red}{[\textbf{#1}: #2]}}
\newcommand{\acomment}[2]{\textcolor{red}{[\textbf{#1}: #2]}}

%% SHORT COMMENT: copy this to make your own inline comment
\newcommand{\flip}[1]{{
  \color{green!50!black}
  \footnotesize
  [\textbf{\textsf{Flip}}: \textsf{#1}]
  }}

%% IN-BOX COMMENTS (for longer comments)
%% -------------------------------------
% Uses: tcolorbox

%% LONG COMMENT: copy this to make your own boxed comment
\newenvironment{flipcomment}
  {
    \begin{tcolorbox}[
      title=Flip Comment,
      fonttitle=\bfseries\sffamily,
      colframe=green!50!black,
      colback=white
      ]
    \small
  }{
    \end{tcolorbox}
  }

%% LONG COMMENT: copy this to make your own boxed comment
\newenvironment{boxedcomment}[1]
  {
    \begin{tcolorbox}[
      title=#1,
      fonttitle=\bfseries\sffamily,
      colframe=green!50!black,
      colback=white
      ]
    \small
  }{
    \end{tcolorbox}
  }


%% ADDING AND REMOVING TEXT
%% ------------------------
%% Analogous to LaTeXdiff-by-hand

\newcommand{\new}[1]{{ 
    \color{green!50!black}\footnotesize
    [\textbf{\textsf{New}}: {#1}]}}


%% REMOVE Environment
%% https://tex.stackexchange.com/a/488582/8094
%% Creates a nolabel environment that strips all labels
%% This is useful to avoid multiple label definitions
%% When marking old versions for deletion
\usepackage{xparse}
\ExplSyntaxOn
\NewDocumentEnvironment{nolabel}{}{
  \cs_set_eq:NN \label \use_none:n
  \cs_set_eq:cN { ltx@label} \use_none:n
}{}
\ExplSyntaxOff 

\newcommand{\remove}[1]{{
  \begin{nolabel} 
    \color{red!50!black}\footnotesize
    [\textbf{\textsf{Removed}}: {#1}]
    \end{nolabel}
    }}
